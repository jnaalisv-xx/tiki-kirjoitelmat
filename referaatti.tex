
%
% 58110 Kandidaatin tutkielma 
% referaatti
% Juho Naalisvaara, juho.naalisvaara@cs.helsinki.fi
%

% rubber: module pdftex
% rubber: path tktl
% rubber: bibtex.stylepath tktl
% rubber: bibtex.path .

\documentclass{tktltiki}
\usepackage{ae,aecompl}
\usepackage{url}
\usepackage{amsfonts}
\usepackage{color}
\usepackage{graphicx}

\begin{document}
\title{Metalingvistinen abstraktio}
\author{Juho Naalisvaara}
\date{\today}
\level{Tieteellisen kirjoittamisen kurssin referaatti}
\maketitle

\onehalfspacing

T�m� kirjoitelma on referaatti \cite{SICP}:n kappaleesta 4.

\section{Metalingvistinen abstraktio}
Ohjelmointi on alati kasvavan kompleksisuuden hallintaa muodostamalla k�ytetyn kielen tarjoamista primitiiveist� suurempia kokonaisuuksia, kuten olioita ja funktiota, ja edelleen kokoamalla niit� suurempiin komponentteihin ja moduuleihin. Itseasiassa n�m� ohjelmoijan m��ritt�m�t uudet primitiivit, muodostavat sovellusaluekohtaisen kielen, jonka kautta on helpompi kuvata ja ratkoa sovellusalueen ongelmia. Eli ohjelmoijata ongelmia ratkoessaan tavallaan suunnittelevat uusia kieli�.

Evaluoija eli tulkki on proseduuri joka osaa evaluoida sy�tteen� saamansa kohdekielen lauseen ja k��nt�� sen semanttisesti vastaaviksi operaatioiksi. T�ss� kirjoituksessa k�sitell��n Lispill� kirjoitettua Lisp-tulkkia. Tulkki joka on kirjoitettu kielell� jota se tulkkaa, sanotaan olevan metasirkulaarinen.

Tulkkauksessa voidaan erottaa kaksi vaihetta.
1.Evaluoitaessa kombinaatioota, evaluoidaan ensin alilausekkeet mink� j�lkeen sovelletaan operaattori-lausekkeen arvoa operandi lausekkeiden arvoihin.
2.Sovellettaessa proseduuria argumenteilleen, evaluoidaan proseduurin vartalo uudessa ymp�rist�ss� miss� proseduurin argumenttien arvot ovat sidottuna muodollisiin parametreihin.

\bibliographystyle{tktl}
\bibliography{lahteet}

\lastpage

\end{document}
