% rubber: module pdftex

\documentclass[t,hyperref={pdfpagelabels=false}]{beamer}
\usepackage{ae,aecompl}
\usepackage[latin1]{inputenc}
\usepackage[finnish]{babel}
\usepackage[T1]{fontenc}

\usecolortheme{rose}
\usefonttheme{structurebold}

\setbeamertemplate{navigation symbols}{}

\title{Funktionaalinen ohjelmointi}
\author{Juho Naalisvaara}
\date{Kirjallisuuspiiri\\
4.11.2009}

\begin{document}

\section{Etusivu}
\begin{frame}
    \titlepage
\end{frame}

% Homman pihvi
% imperatiivinen ohjelmointi
% miten funktionaalinen ohjelmointi vertautuu siihen, miten eroaa
% miksi



\section{Esimerkkej�}
\subsection{Listoja}
\begin{frame}
    \frametitle{Listoja}
    \begin{itemize}
    \item Ensimm�inen kohta
    \item Toinen kohta
    \item \alert{T�rke�} asia
    \end{itemize}
\end{frame}

\subsection{Laatikoita}
\begin{frame}
    \frametitle{Laatikoita}
    \begin{block}{Esimerkki 1}
    Teksti�\\
    teksti�\\
    teksti�
    \end{block}
    \begin{block}{Esimerkki 2}
    Teksti�\\
    teksti�\\
    teksti�
    \end{block}
\end{frame}

\subsection{Palstoja}
\begin{frame}
    \frametitle{Palstoja}
    \begin{columns}[onlytextwidth,T]
        \column{5cm}
        Teksti�\\
        teksti�\\
        teksti�
        \column{5cm}
        Teksti�\\
        teksti�\\
        teksti�
    \end{columns}
\end{frame}

%\subsection{Kuva}
%\begin{frame}
%    \frametitle{Kuva}
%    \begin{center}
%        \input{esimerkkikuva.pdf_t}
%    \end{center}
%\end{frame}


\end{document}
